%-------------------------
% Resume in Latex
% Author  : Mohit Sharma
% License : MIT
%------------------------

\documentclass[letterpaper,10.5pt]{article}

\usepackage{latexsym}
\usepackage[empty]{fullpage}
\usepackage{titlesec}
\usepackage{marvosym}
\usepackage[usenames,dvipsnames]{color}
\usepackage{verbatim}
\usepackage{enumitem}
\usepackage[hidelinks]{hyperref}
\usepackage{fancyhdr}
\usepackage{xcolor}
\usepackage{siunitx}


\pagestyle{fancy}
\fancyhf{} % clear all header and footer fields
\fancyfoot{}
\renewcommand{\headrulewidth}{0pt}
\renewcommand{\footrulewidth}{0pt}

% Adjust margins
\addtolength{\oddsidemargin}{-0.5in}
\addtolength{\evensidemargin}{-0.5in}
\addtolength{\textwidth}{1in}
\addtolength{\topmargin}{-0.5in}
\addtolength{\textheight}{1.0in}

\urlstyle{same}

\raggedbottom
\raggedright
\setlength{\tabcolsep}{0in}

% Sections formatting
\titleformat{\section}{
\vspace{-4pt}\scshape\raggedright\large
}{}{0em}{}[\color{orange} \titlerule \vspace{-5pt}]

%-------------------------
% Custom commands
\newcommand{\resumeItem}[2]{
\item\small{
\textbf{#1}{: #2 \vspace{-2pt}}
}
}

\newcommand{\resumePubItem}[3]{
\item\small{
\textbf{#1 \null\hfill{#2}}{#3} \vspace{-2pt}
}
}

\newcommand{\resumeSubheading}[4]{
\vspace{-1pt}\item
\begin{tabular*}{0.97\textwidth}[t]{l@{\extracolsep{\fill}}r}
    \textbf{#1}       & #2                 \\
    \textit{\small#3} & \textit{\small #4} \\
\end{tabular*}\vspace{-5pt}
}

\newcommand{\resumeSubItem}[2]{\resumeItem{#1}{#2}\vspace{-4pt}}

\newcommand{\resumePublicationItem}[3]{\resumePubItem{#1}{#2}\newline{#3}\vspace{-4pt}}

\renewcommand{\labelitemii}{$\circ$}

\newcommand{\resumeSubHeadingListStart}{\begin{itemize}[leftmargin=*]}
\newcommand{\resumeSubHeadingListEnd}{\end{itemize}}
\newcommand{\resumeItemListStart}{\begin{itemize}}
\newcommand{\resumeItemListEnd}{\end{itemize}\vspace{-5pt}}

\newcommand{\resumeTilde}{\raise.17ex\hbox{$\scriptstyle\sim$}}
%-------------------------------------------
%%%%%%  CV STARTS HERE  %%%%%%%%%%%%%%%%%%%%%%%%%%%%


\begin{document}

%----------HEADING-----------------
\begin{tabular*}{\textwidth}{l@{\extracolsep{\fill}}r}
\textbf{\href{https://sharmamohit.com/} {\color[HTML]{DF691A} Mohit Sharma}} & \color[HTML]{DF691A} Email : \href{mailto:mohitsharma44@gmail.com}{mohitsharma44@gmail.com}\\
\href{https://sharmamohit.com/}{https://www.sharmamohit.com}  & Mobile : +1-778-587-6241\\
\end{tabular*}


%-----------SUMMARY-------------------
\section{\color[HTML]{DF691A} Summary}
Innovative DevOps engineer with a strong Linux background and 6+ years of experience designing, automating and managing mission critical infrastructure deployments by leveraging configuration management tools and other DevOps processes. Expert in scripting using python with an emphasis on real-time, high speed data pipelines and distributed computing across networks.


%-----------EXPERIENCE-----------------
\section{\color[HTML]{DF691A} Experience}
\resumeSubHeadingListStart

\resumeSubheading
{Workday}{Victoria, BC, Canada}
{Software Development Engineer II, DevOps}{May 2019 - Present}
\resumeItemListStart
\resumeItem{Development and Operations}{Created an automated AWS deployment pipeline to deploy microservice application cross account and cross region
reducing operations toil by several hours.}
\newline
{Actively manage, improve, and monitor infrastructure resources in DC and on AWS including but not limited to
EC2, ECS, Route53, S3, RDS, Lambda, ES etc. using tools like terraform, wavefront, ELK stack and slackbot.}
\newline
{Writing ansible roles to help manage the hosts and perform application deployment.}
\newline
{Writing Jenkinsfiles and improving shared libraries for automated build and deployment of several applications and services using Jenkins.}
\newline
{Migrating application services to an in-house flavor of Kubernetes platform.}
\resumeItem{Knowledge Share}{Spearheaded an initiative between operations and multiple dev teams to empower the developers with the knowledge
they would need to take ownership of their services.}
\resumeItemListEnd

\resumeSubheading
{NYU CUSP}{Brooklyn, NY, USA}
{Associate Research Scientist}{May 2015 - May 2019}

\textit{\small{Assistant Research Scientist}} \null\hfill \textit{\small{June 2014 - May 2015 .}}
\resumeItemListStart
\resumeItem{Dockkeeper}
{Developed a scalable and secure container scheduling and monitoring tool that leverages the docker ecosystem and prometheus for provisioning services on physical hosts. This helped eliminate the VM license fees of over \$35,000 per year and optimizing the efficiency of hosts by over 55\%. \\Deployed a multi-node kubernetes cluster for exposing load-balanced web applications on the web.}
\resumeItem{UOInfra}
{Architected the NYU/CUSP Urban Observatory's multi-site physical infrastructure consisting of multiple dense compute and storage nodes comprising of over half a petabyte of storage space, provisioned for multi-user mini-HPC environments, using Ansible and Packer.\\
Deployed a 27 screen vizwall using a cluster of networked raspberry Pi's enabling researchers to interpret their visualizable data, while keeping the whole price to 1/8 th of that of a commercial solution.}
\resumeItem{SONYC}
{Developed a secure machine critical IoT platform and implemented CI/CD framework for deploying and maintaining over 100 urban noise monitoring sensors in NYC. This project has won the \href{https://www.nsf.gov/awardsearch/showAward?AWD_ID=1544753}{\$ $4.6$ Million} CPS frontier award from NSF. An innovative lifeline beacon based approach helped reduce the time to revive sensors in the field from \resumeTilde2 week per node to less than 1 hour per node, improving the efficiency of the team and the sensor network.}
\resumeItemListEnd
\resumeSubHeadingListEnd

%-----------EDUCATION-----------------
\section{\color[HTML]{DF691A} Education}
\resumeSubHeadingListStart
\resumeSubheading
{NYU Polytechnic School of Engineering}{Brooklyn, NY, USA}
{Master of Science in Telecommunication Networks}{Aug. 2012 -- May. 2014}

\textit{\small{Thesis - \href{https://sharmamohit.com/project/citysynth/}{CitySynth: Imaging with a Network of Devices}}}
%\resumeSubheading
%  {Birla Institute of Technology and Science}{Pilani, India}
%  {Bachelor of Engineering in Electrical and Electronics;  GPA: 3.66 (9.15/10.0)}{Aug. 2008 -- July. 2012}
\resumeSubHeadingListEnd


%-----------CERTIFICATIONS-----------
\section{\color[HTML]{DF691A} Certifications}
\resumeSubHeadingListStart
\resumeSubItem{CKA Certified Kubernetes Administrator} \null\hfill \textit{\small{August 2020 - August 2023}}
\newline
\newline
{Certificate Id: \href{https://www.youracclaim.com/badges/ce11744e-269f-459d-836a-9ce4c11ff2c9}{LF-bu4edd0v34}}
\resumeSubHeadingListEnd

%-----------PROJECTS-----------------
\section{\color[HTML]{DF691A} Projects}
\resumeSubHeadingListStart
\resumeSubItem{CUIC}
{Open source python library for interfacing with GigE vision broadband, thermographic and hyperspectral cameras using advanced message queuing protocol to acquire images and perform pre-processing on-the-fly.}
\resumeSubItem{UCSLHUB}
{Developed a resilient and scalable back-end infrastructure using docker swarm, jupyterhub and keycloak for hosting the CUSP's UCSL bootcamp which will be accessed by hundreds of students every year.}
\resumeSubItem{HOMELAB}
{Running homelab with ansible managed VMWare ESXI and VSphere instances behind a Pfsense firewall that is used for testing and development including emulation of distributed datacenters for UOInfra.}
\resumeSubHeadingListEnd

\section{\color[HTML]{DF691A} Publications, Teaching Experience and more..}
\resumeSubHeadingListStart
\resumeItem{View it online}{\href{https://sharmamohit.com/work/}{https://sharmamohit.com/}}
\newline
\resumeSubHeadingListEnd
%%-----------PUBLICATIONS-----------------
%\section{\color[HTML]{DF691A} Publications}
%\resumeSubHeadingListStart
%\resumePublicationItem{\href{https://ieeexplore.ieee.org/document/8646419}{Persistent Hyperspectral Observations of the Urban Lightscape}}{\textit{IEEE GlobalSIP}, 2018}{Training a supervised classifier to automatically determine location of light sources on persistent hyperspectral imaging of the New York City urban lightscape, with \resumeTilde 7.2 x \num{e-4} \SI{}{\micro\metre} spectral resolution, surveyed over 25 consecutive summer nights over a 6 minute time resolution using Dockkeeper infrastructure.}
%\resumePublicationItem{\href{http://www.mdpi.com/1424-8220/16/12/2047/html}{A Hyperspectral Survey of New York City Lighting Technology}, 2016}{\textit{Sensors}, 16, 12}
%{Using a scanning, single channel spectrograph to identify the lighting technologies in use in the NYC}
%\resumePublicationItem{\href{http://dl.acm.org/citation.cfm?id=2993570}{Hypertemporal imaging of NYC Grid Dynamics}, 2016}{\textit{BuildSys '16}}
%{Demonstrating the concept of capturing the 120 Hz flicker of lights across a NYC skyline as a proxy to indicate the health of distribution transformers}
%\resumePublicationItem{\href{http://www.sciencedirect.com/science/article/pii/S0306437915001167}{Dynamics of Urban Lightscape}, 2015}{\textit{Information System}, 54, 115}
%{Using a network of cameras to understand \textit{the pulse of the city}}
%\resumeSubHeadingListEnd
%
%%-----------Teaching-----------------
%\section{\color[HTML]{DF691A} Teaching Experience}
%\resumeSubHeadingListStart
%\resumePublicationItem{ \href{https://sharmamohit.com/\#teaching}{\textbf{Urban Computing Skills Lab}} at NYU}{\textit{2014 - 2019}}
%{Instructor for summer boot camp course on introduction to Python and SciPy packages.}
%\resumePublicationItem{ \href{https://sharmamohit.com/\#teaching}{\textbf{Advanced Topics in Urban Informatics}} at NYU}{\textit{2016, 2017, 2019}}
%{Instructor for a 3 week intensive course on topics including Wireless Sensor Networks, IoT and Microservices}
%%\resumePublicationItem {CUSP City Challenge Week}{\textit{2016}}{Advised 14 graduate students on a 2 day challenge of Analyzing Crime in NYC}
%\resumePublicationItem{Advised NYU/CUSP graduate student Denis Khryashchev}{\textit{2015 - 2016}}{project: \href{https://www.overleaf.com/read/ssrzkqkznpkw}{\textit{Social Pattern Detection by scanning GSM downlink spectrum}}}
%\resumeSubHeadingListEnd

\end{document}
